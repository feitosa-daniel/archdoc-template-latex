\chapter{\LaTeX{} Formatting}

There are some great resources to help with \LaTeX,
so this guide will not include general instructions.
For those, consult the Latex Wikibook\footnote{\url{https://en.wikibooks.org/wiki/LaTeX}}
and the excellent guidelines by Spinellis~\cite{spinellis:latexadvice}.

That said, there are some small pointers that are particular to this template.
These are mentioned in the subsection below.

\section{References}
\label{sec:a9:References}

\noindent
\textbf{Indirect citation}

These look like ``\verb|... in related work~\cite{source}|'' and generate something like ``\dots in related work~[2].''
Do not forget the ``\verb|~|'' before ``\verb|\cite|'' to not let the next line start with ``\verb|[2]|''.

\vspace{0.5cm}
\noindent
\textbf{Direct citation}

These look like ``\texttt{according to Author et} \verb|al.~\cite{source}|'' and generate something like ``according to Author et al. [2].''
Only add ``et al.'' if the reference is by three or more people.

\section{Tables}
\label{sec:a9:latex_tables}

Avoid cumbersome table formatting (e.g. using vertical lines).
In general, simpler is better.
\autoref{tab:a9:simpleexample} and \autoref{tab:a9:multicolexample} are some examples.
Also, look at the tables' source for code formatting tips.

\begin{table} [h]
    \caption{Example of a simple table (based on~\cite{spinellis:latexadvice})}
    \label{tab:a9:simpleexample}
    
    \noindent
    \centering
    \footnotesize%if needed

    % put the header on a \parbox to center and break lines
    \newcommand{\hb}[2]{\parbox[c][0.8cm][c]{#1}{\centering #2}}
    
    \begin{tabular}{lccccc}
        \toprule
        % Column labels (using separate lines to align label with data)
        \hb{2cm}{Command\\and options}
                & \hb{2cm}{Input\\requirements}
                        & Matched
                                & Connected
                                        & Matched
                                                & Connected \\
        \midrule
        tr -cs  & 1     & X    & --    & V    & 1 \\
        sort w  & 0     &       & --    & X    & 0 \\
        fmt     & 1     & X    & --    & V    & 1 \\
        tr A-Z  & 1     & V    & 1     &       & 1 \\
        sort -u & fmt   & X    & --    & V    & 1 \\
        \bottomrule
    \end{tabular}
\end{table}


\begin{table} [h]
    \caption{Example of multi-column table}
    \label{tab:a9:multicolexample}
    
    \noindent
    \centering
    
    \begin{tabular}{lSSSS}
        \toprule
        {}                & \multicolumn{2}{c}{\textbf{Group 1}} 
                                                    & \multicolumn{2}{c}{\textbf{Group 2}} \\
                            \cmidrule(r){2-3}         \cmidrule(r){4-5}
        {}                & {Average}  & {Max}      & {Average}   & {Max} \\
        \midrule
        \textbf{Factor 1} &   853      &  2443      &   1760      &   3799      \\
        \textbf{Factor 2} & 14983      & 74658      & 110881      & 316552      \\
        \textbf{Factor 3} &    18      &    85      &     89      &    319      \\
        \textbf{Rate}     &    51.91\% &    86.34\% &     44.44\% &     81.08\% \\
        \bottomrule
    \end{tabular}
\end{table}

\section{Figures}
\label{sec:a9:latex_figures}

The main tips in a nutshell are as follows.
\begin{itemize}
    \item Put all your figures inside the \texttt{figs} folder,
          and use subfolders to organize them per chapter.
    \item Prefer vectorized images (e.g. PS or PDF).
          Bear in mind that saving a PNG or JPG as PDF
          will not vectorize your image.
\end{itemize}

See \autoref{fig:a9:simpleexample} for a simple example.
\autoref{fig:a9:allauthors} shows an example using sub-figures.
Notice that you can also cite them individually
(see \autoref{fig:a9:auth1}, \autoref{fig:a9:auth2} and \autoref{fig:a9:auth3}).

\begin{figure}
    \centering
    \includegraphics[width=.2\textwidth]{figs/author}
    \caption{A figure example}
    \label{fig:a9:simpleexample}
\end{figure}

\begin{figure}
\centering
\subcaptionbox{Author 1\label{fig:a9:auth1}}[0.2\textwidth]{%
    \includegraphics[width=0.2\textwidth]{figs/author}}%
\hspace{0.1\textwidth} % seperation
\subcaptionbox{Author 2\label{fig:a9:auth2}}[0.2\textwidth]{%
    \includegraphics[width=0.2\textwidth]{figs/author}}%
\hspace{0.1\textwidth} % seperation
\subcaptionbox{Author 3\label{fig:a9:auth3}}[0.2\textwidth]{%
    \includegraphics[width=0.2\textwidth]{figs/author}}%
\caption{All authors}
\label{fig:a9:allauthors}
\end{figure}